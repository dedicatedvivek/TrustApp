\chapter{Literature Survey}

%In many DSP algorithms, the multiplier lies in the critical delay path and ultimately determines the performance of algorithm. The speed of multiplication operation is of great importance in DSP as well as in general processor. In the past multiplication was implemented generally with a sequence of addition, subtraction and shift operations. There have been many algorithms proposals in literature to perform multiplication, each offering different advantages and having tradeoff in terms of speed, circuit complexity, area and power consumption.
%
%Digital multipliers are the core components of all the digital signal processors (DSPs) and the speed of the DSP is largely determined by the speed of its multipliers. Hardware implementation of the add-and-shift multiplication algorithm is faster than software synthesis but nevertheless, the demand for higher speed result in various multiplier architectures to push the limit. A plain add-and-shift algorithm is slow in hardware because as each additional partial product is summed, a carry must be propagated from the least significant bit to the most significant bit. This carry propagation is the main speed bottleneck and several techniques have evolved that target the optimization of the carry propagation delay.

Two most common multiplication algorithms followed in the digital hardware are array multiplication algorithm and Booth multiplication algorithm.
In an array multiplier multiplication of two binary numbers can be obtained with one micro-operation by using a combinational circuit that forms the product bits all at once thus making it a fast way of multiplying two numbers since the only delay is the time for the signals to propagate through the gates that form the multiplication array. However, an array multiplier requires a large no gates and for this reason it is less economical \cite{r4}.

Booth algorithm is another powerful algorithm for multiplication. This algorithm is a method that will reduce the number of multiplicand multiples. For a given range of numbers to be represented, a higher representation radix leads to fewer digits. Since a k bit binary number can be interpreted as k/2 digit radix – 4 number, a k/3 digit radix – 8 number and so on it can deal with more than one bit of the multiplier in each cycle by using high radix multiplication. Most of the multiplication algorithms for high speed implementation are based on Booth encoding algorithm and its modifications \cite{r2}.
%The main disadvantage of the add and shift method is that it is slow and that any temporal correlation between successive data words will be removed due to the time multiplexed nature of the computation which has an adverse effect on the power consumption of the implementation.

%In the Wallace tree method, three bit signals are passed to a one bit full adder (3W) which is called a three input Wallace tree circuit , and the output signal (sum signal ) is supplied to the next stage full adder of the same bit, and the carry output signal thereof is passed to the next stage full adder of the same no of bit located at a one bit higher position. In the Wallace tree method, the circuit layout is not easy although the speed of the operation is high since the circuit is quite irregular.

In \cite{r3} an algorithm for high-speed, two's complement, m-bit by n-bit parallel array multiplication is described. The two's complement multiplication is converted to an equivalent parallel array addition problem in which each partial product bit is the AND of a multiplier bit and a multiplicand bit, and the signs of all the partial product bits are positive. Main disadvantage of this algorithm was the need for the complements of each multiplier and multiplicand bit in forming the partial products bits.

Many have proposed the use of Vedic mathematics methods for multiplication of unsigned binary numbers. In \cite{r6} a NxN bit parallel overlay multiplier architecture for high speed DSP operations is proposed. The architecture is based on the Urdhva Tiryakbhyam – the Vertical and Crosswise algorithm of Vedic Mathematics. The whole multiplication operation is decomposed into 4 x 4 bit multiplication modules. The 4 x 4 multiplication modules are implemented using array and booth multipliers and a considerable improvement in the speed is achieved.

%A Multiply and Accumulate (MAC) architecture using Vedic Mathematics is proposed by Thapliyal H. and Srinivas M.B. (2004), suited for high speed applications. In the proposed architecture all bits of operands (multiplier and multiplicand) and accumulator are presented in parallel. This
%multiplier concurrently adds the partial product bits generated with the accumulator bits and improves speed.

A new multiplier and square architecture is proposed in \cite{r4}, based on the algorithm of Ancient Indian Vedic Mathematics, for low power and high speed applications. It is based on generating all partial products and their sums in single step. It has been shown that as the number of bits in the multiplier increases the Vedic Multiplier has superior scalability over conventional multiplier methods. Similarly for the square architecture the gate delay and area reduces by 50 percentage \cite{r4}.

In \cite{r5} design and implementation of a novel high speed signed multiplier based on vedic mathematics is proposed. In which multiplication is done by vedic multiplier and partial product addition is done by ripple carry adder.


%Another improvement in the multiplier is by reducing the numbers of partial products generated. The Booth recording multiplier is one such multiplier; it scans the three bits at a time to reduce the number of partial products [24]. These three bits are: the two bit from the present pair; and a third bit from the high order bit of an adjacent lower order pair. After examining each triplet of bits, the triplets are converted by Booth logic into a set of five control signals used by the adder cells in the array to control the operations performed by the adder cells.
%
%The method of Booth recording reduces the numbers of adders and hence the delay required to produce the partial sums by examining three bits at a
%time. The high performance of Booth multiplier comes with the drawback of power consumption. The reason for this is the large number of adder cells (15 cells for 8 rows-120 core cells) that consume power [24]. The conclusion is that the current methodology of multiplication leads to more consumption of power and reduction in efficiency.

Thus many attempts have been reported in literature about improvement in multipliers that have the least number of gate delays and consume the least amount of chip area. So there is a need for an improved multiplier architecture that has the simple design advantages, but which does not suffer the excessive delays associated with conventional multiplier structures. This work presents a signed multiplier architecture based on the ancient Indian Vedic mathematics \cite{r1} sutra (formula) called Urdhva Tiryakbhyam ( Vertically and Cross wise) which is traditionally used for decimal system in ancient India. The designs of the multiplier is considerably faster than existing multipliers reported previously in the literature. It is demonstrated that this design is quite efficient in terms of speed.

%This paper proposes a novel multiplier and square architecture providing the solution of the aforesaid problems adopting the sutra (formula) of Vedic Mathematics called Urdhva Tiryakbhyam (Vertically and Cross wise)[25,26,27]. The designs of the multiplier and square are considerably faster than existing multipliers reported previously in the literature. It is demonstrated that this design is quite efficient in terms of silicon area/speed.
%The common multiplication method is “add and shift” algorithm. In parallel multipliers number of partial products to be added is the main parameter that determines the performance of the multiplier. To reduce the number of partial products to be added, Modified Booth algorithm
%is one of the most popular algorithms. To achieve speed improvements Wallace Tree algorithm can be used to reduce the number of sequential adding stages. Further by combining both Modified Booth algorithm and Wallace Tree technique we can see advantage of both algorithms in one multiplier. However with increasing parallelism, the amount of shifts between the partial products and intermediate sums to be added will increase which may result in reduced speed, increase in silicon area due to irregularity of structure and also increased power consumption due to increase in interconnect resulting from complex routing. On the other hand “serial-parallel” multipliers compromise speed to achieve better performance for area and power consumption. The selection of a parallel or serial multiplier actually depends on the nature of application.












%
%
%
%%Traditiovnally, diggfital signal processing (DSP) algorithms are
%%implemented using general-purpose (programmable) D
%%in section \ref{organization},we haveeee further,subsection\ref{partt1}
%%\begin{tabular}{|c|c|}
%%                                                                         \hline
%%                                                                         % after \\: \hline or \cline{col1-col2} \cline{col3-col4} ...
%%                                                                         x & y \\
%%                                                                         1 & 2 \\
%%                                                                         \hline
%%                                                                       \end{tabular}
