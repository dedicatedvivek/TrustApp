\documentclass{beamer}
%\mode<presentation>{\usepackage{beamerthemesplit}}

%\usepackage{beamerthemebars}
\usepackage{amsmath}
\renewcommand{\baselinestretch}{1.2}
\usepackage{cite}
\usepackage{url}
\usepackage{longtable}
%\usepackage[dvips]{graphicx,color}
%\usepackage{makeidx}
\usepackage{nomencl}
\usepackage{amssymb}
%\usepackage{psfig}
%\usepackage{epsfig}
\usepackage{graphicx}
\usepackage{amssymb}
\usepackage{multicol}
\usepackage[bottom]{footmisc}
\usepackage{subfigure}
\usepackage[OT2,OT1]{fontenc}
\newcommand{\imsize}{3in}

\useinnertheme{rounded}
\usecolortheme{whale}
%\usecolortheme{orchid}
\useoutertheme{infolines}
%\useoutertheme{shadow}

%change your title
\title{Analysis of Electrical Circuits and Mechanical
Systems with Fractional-order Elements: A
Systems Theory Approach }
\subtitle{Dissertation-I}
%\institute{\normalsize{IIT Bombay}}
%\setbeamercolor{title}{fg=white,bg=black}

\begin{document}
\author[Yogesh Gulekar] {By \\ \vspace{0.05in} \textbf{Yogesh M. Gulekar } \\ \vspace{0.01in} {Under the Guidance of} \\ \textbf{Prof. M. D. Patil}\\
\textcolor{black}{ \\ Department of Electronics Engineering} \\ \textcolor{black}{Ramrao Adik Institute of Technology,\\ Nerul, Navi Mumbai } \\
}

\begin{frame}
\begin{center}
\end{center}
\titlepage
\end{frame}


%Slide which includes bullated items having numerical no.
\begin{frame}
\frametitle{Outline}
\begin{enumerate}
\item {Introduction.}
\item {What is Fractional Calculus.}
\item {Fractional Order (FO) Elements. }
\item {Circuit With FO Elements.}
\item {Fractional Order Transfer Function.}
\item {Stability Of FO System.}
\item {Problem Definition.}
\end{enumerate}
\end{frame}

%Slide which includes bullated items having numerical no.
\begin{frame}
\frametitle{Introduction}
\begin{itemize}
\item {Integer Order System and its limitations.}
\item {Fractional Order System and its approach. }
\item {Basics of Fractional Order System.}
\item {FO elements and its applications.}
\item {Problem Definition.}
\item {Conclusion.}
\end{itemize}
\end{frame}

%Slide which includes bullated items having numerical no.
\begin{frame}
\frametitle{Fractional Calculus}
Real order generalization of fractional calculus
\begin{equation}
{D}^\alpha = \mbox{\Huge\{}
               \begin{array}{cc}
                 \frac{d^\alpha}{dt^\alpha} & \alpha > 0 \\
                 1 & \alpha = 0 \\
                 {\int_a}^t {(d\tau)^\alpha} & \alpha < 0 \\
               \end{array}
\end{equation}
%
with $\alpha\in \mathcal{R}$.
\end{frame}

\begin{frame}
\frametitle{ Riemann-Liouville:}
%
\textbf{Integral:}
%
\begin{equation}\label{CurrentVoltageRelationshipForFractionalCapacitor}
{{J_c}^\alpha}f(t) = \frac{1}{\Gamma\alpha }\int \frac{f(\tau)}{{(t-\tau) }^{1-\alpha}}d\tau
\end{equation}
%
\textbf{Derivative:}
%
\begin{equation}\label{CurrentVoltageRelationshipForFractionalCapacitor}
{{D}^\alpha}f(t) = \frac{d^m}{dt^m}[\frac{1}{\Gamma (m- \alpha) }{\int_a}^t\frac{f(\tau)}{{(t-\tau) }^{\alpha+1-m}}d\tau], m\in Z^{+}, m-1 < \alpha\leq m.
\end{equation}
\end{frame}

\begin{frame}
\frametitle {Grunwald-Letnikov:}
%
\textbf{Integral:}
%
\begin{equation}\label{CurrentVoltageRelationshipForFractionalCapacitor}
{{D}^{- \alpha}} = \lim _{h\longrightarrow0} h^{\alpha}\sum_{m=0}^{(t-a)/h}\frac{\Gamma (\alpha+m)}{m!\Gamma (\alpha)}f(t-mh),
\end{equation}
%
\textbf{Derivative:}
%
\begin{equation}\label{CurrentVoltageRelationshipForFractionalCapacitor}
{{D}^{ \alpha}} = \lim _{h\longrightarrow0}\frac {1}{h^{\alpha}}\sum_{m=0}^{(t-a)/h}(-1)^{m}\frac{\Gamma (\alpha+1)}{m!\Gamma(\alpha-m+1)}f(t-mh),
\end{equation}
\end{frame}

\begin{frame}
\frametitle{Caputo:}
%
\begin{equation}\label{CurrentVoltageRelationshipForFractionalCapacitor}
{{D}^\alpha}f(t) = \frac{1}{\Gamma (m- \alpha) }{\int_0}^t \frac{f^{m}(\tau)}{{(t-\tau) }^{\alpha+1-m}}d\tau,
\end{equation}
\end{frame}

\begin{frame}
\begin{itemize}
\item {Inductor or lossy coil have hysteresis and eddy current losses.}
\item {Capacitors produces dielectric losses.}
\item {FO inductors or FO capacitors  yield a more accurate description.}
\end{itemize}
\end{frame}

%Slide which includes figure
\begin{frame}
\frametitle{FO Elements} Schematic of
FO Capacitor
\begin{figure}
\begin{center}
\includegraphics[width= 4 in]{FOmodelRpztnOfossyC.pdf}
\end{center}
%\caption{The two degree-of-freedom structure in QFT} \label{tdof}
\end{figure}
\end{frame}



\begin{frame}
The voltage-current
relationships for an FO capacitor are
%
\begin{equation}\label{CurrentVoltageRelationshipForFractionalCapacitor}
i_c(t) = C_\alpha \frac{d^\alpha}{dt^\alpha}v_c(t),
\end{equation}
%
or
%
\begin{equation}\label{VoltageCurrentRelationshipForFractionalCapacitor}
v_c(t) = \frac{1}{C_\alpha} \textmd{ }{_0J_t}^\alpha i_c(t),
\end{equation}
%
where the constant $0<\alpha<1, \alpha \in \mathbf{R}$ is a
measure of the losses in
the capacitor.
%
\begin{equation}
j^{-\alpha} = \cos(\alpha\pi/2) - j\sin(\alpha\pi/2)= e^{-j\alpha\pi/2}
\end{equation}
\end{frame}

%Slide which includes figure
\begin{frame}
\frametitle{FO Elements} Phase Relationship Between V and I for FO capacitor
\begin{figure}
\begin{center}
\includegraphics[width= 2 in]{PhaseRelationBtwnVandIforFOC.pdf}
\end{center}
%\caption{The two degree-of-freedom structure in QFT} \label{tdof}
\end{figure}
\end{frame}

%Slide which includes figure
\begin{frame}
\frametitle{FO Elements} Schematic of FO Inductor
\begin{figure}
\begin{center}
\includegraphics[width= 4 in]{FOmodelofL.pdf}
\end{center}
%\caption{The two degree-of-freedom structure in QFT} \label{tdof}
\end{figure}
\end{frame}



\begin{frame}
For an FO inductor we have,
%
\begin{equation}
V_L(t) = L_\beta \frac{d^\beta}{dt^\beta}i_L(t)
\end{equation}
%
where the fractional power $\beta$ is related
to the phenomenon of proximity effect. In frequency domain,
%
\begin{equation}
V_L(s)= s^\beta L_\beta I_L(s)
\end{equation}
%
\begin{eqnarray}
j= \cos(\pi/2) + j\sin(\pi/2)= e^{j\pi/2}\\
j^\beta= \cos(\beta\pi/2) + j\sin(\beta\pi/2)= e^{j\beta\pi/2}
\end{eqnarray}
\end{frame}
%
%Slide which includes figure
\begin{frame}
\frametitle{FO Elements} Phase Relationship Between V and I for FO inductor
\begin{figure}
\begin{center}
\includegraphics[width= 2 in]{PhasorDiagForFOL.pdf}
\end{center}
%\caption{The two degree-of-freedom structure in QFT} \label{tdof}
\end{figure}
\end{frame}

%Slide which includes bullated items having numerical no.
\begin{frame}
\frametitle{Stability of FO System}
\begin{itemize}
\item {Fractional Domain}
\item {Mapping of S-plane to F-plane. }
\item { $\alpha$ increases the stable F-domain regions decreases.}
\item {Stability decreases for $\alpha > $2 }
\end{itemize}
\end{frame}


%Slide which includes figure
\begin{frame}
\frametitle{Stability of FO}
\begin{figure}
\begin{center}
\includegraphics[width= 2 in]{stability2.pdf}
\end{center}
%\caption{The two degree-of-freedom structure in QFT} \label{tdof}
\end{figure}
\end{frame}

%Slide which includes figure
\begin{frame}
\frametitle{Problem Definition}
\begin{figure}
\begin{center}
\includegraphics[width= 2 in]{R-Lckt.pdf}
\end{center}
%\caption{The two degree-of-freedom structure in QFT} \label{tdof}
\end{figure}
\end{frame}

\begin{frame}
\frametitle{Transfer Function of FO}
\begin{equation}\
\frac{I(s)}{V(s)} = \frac{\frac{1}{L_\beta}}{S^\beta + \frac {R}{L_\beta}}.
\end{equation}
\end{frame}

%Slide which includes bullated items having numerical no.
\begin{frame}
\frametitle{Future Work}
\begin{itemize}
\item {Analysis of electrical circuits and mechanical systems with fractional-order elements.}
\item {Development of various linear models. }
\item {Stability analysis using Riemann sheet concept.}
\item {Analysis of controllability, observability, reacheability, and solution of
the state-equation.}
\item {Design of linear compensators and controllers.}
\item {Development of equivalent relationship between FO electrical and mechanical
systems.}
\end{itemize}
\end{frame}

%Slide which includes last slides
\begin{frame}
\centerline{\textbf{\begin{Huge}Thank You...\end{Huge}}}
\end{frame}
\end{document}

