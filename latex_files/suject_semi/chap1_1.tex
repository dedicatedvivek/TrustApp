\chapter{\textbf{Introduction}}
This report introduces Belief Propagation Techniques and presents some of the  applications.
\\ Belief Propagation Techniques approach uses the degree of a person's belief that an event will occur,rather than the actual probability  of the event will occur.
Belief probabilities are  properties of person's belief not the event.
\\ A Probabilistic model is a dependency model in which relationship between  each of the variable is captured.Dependency models  are represent by graphical methods.
\\ Belief Propagation Technique is used to perform inference on graphical  model  such as factor graphs which calculates marginal distribution for each unobserved node
conditional on observed node.
The graphical representation used for belief propagation Techniques are two types ,they are Markov  Random Fields and  Bayesian Networks.\cite{Hans Andrea Loeliger}
\\Belief Propagation Techniques  are used for performing inference on graphical models such as Bayesian Network and Markov Random Field\cite{Judea Pearl}
\\Bayesian network uses the directed graphs where the directions of the arrows permits distinguish genuine dependencies for spurious dependencies induced by hypothetical observations
\\In Markov Random  Fields,the network topology was presumed to be given and problem was to characterize the probabilistic  behavior  of a system complying with dependencies prescribed by network.
\\Belief Propagation Techniques are used in  Artificial Intelligence,Signal Processing and  Digital Communication.
\\Some of the applications where Markov  Random Fields and  Bayesian Networks  methods are used  either as  optimization tool or for implementation  are discussed in chapter 3

\begin{itemize}
 \item \textbf{Efficient Belief Propagation for Early Vision.}The method used in this application is Markov Random  Fields on the Belief Propagation to solve problems for low level vision.                                                                 \cite{Pedro}
  
  \item \textbf{Efficient Loopy Belief Propagation using the Four Color Theorem} Four Color Theorem is based on Max-Product Belief Propagation Technique can be used in solving Markov Random  Fields problems where energy is minimized.                     \cite{Radu}
 \item \textbf{Markov Network-based Unified Classifier for Face Identification} Markov Random  Fields is used  for face recognition for one to many identification task                        \cite{Wonjun}
 \\\\\ \item A optimization technique which carries prioriy based message scheduling and dynamic label pruning are used for Markov Random  Fields with large discrete state space is used in this application of Belief Propagation Technique \textbf{Image Completion Using Efficient Belief Propagation Via Priority Scheduling and Dynamic Pruning}           \cite{Nikos}
\item  A new stereo matching algorithm partions an image to block and optimizes with Belief propagation Technique used in this application \textbf{Low Memory Cost Block-Based Belief Propagation For Stereo Correspondence}\cite{yuchang}
   \item A Parallel Techniques are used for implementation Belief Propagation in a acyclic factor graph are the methods used in this research \textbf{ Task Parallel Implementation of Belief Propagation in Factor Graphs}\cite{Nam Ma}
   \item A  Tile based Belief Propagation splits the Markov random field into many tiles and constructing technique is based on observation that many hypotheses is used to construct the messages are repetitive   are method  used in this research work \textbf{Hardware-Efficient Belief Propagation}\cite{Chia-Kailiang}
 \item A method of Particle Belief Propagation (PBP) is used in \textbf{PatchMatch Belief Propagation for
Correspondence Field Estimation.}\cite{besse}


 \item Continuous time  Bayesian network method is used in \textbf{Learning continuous time Bayesian network classifiers}\cite{Daniele}
\end{itemize}
The organization of report is as follows Introduction in \textbf{Chapter-1},Mathematics related to Belief Propagation Techniques is in \textbf{Chapter-2},Different applications related to Belief Propagation Techniques are in \textbf{Chapter-3},conclusion in \textbf{Chapter-4.}











