\documentclass{beamer}
%\mode<presentation>{\usepackage{beamerthemesplit}}

%\usepackage{beamerthemebars}
\usepackage{amsmath}
\renewcommand{\baselinestretch}{1.2}
\usepackage{cite}
\usepackage{url}
\usepackage{longtable}
%\usepackage[dvips]{graphicx,color}
%\usepackage{makeidx}
\usepackage{nomencl}
\usepackage{amssymb}
%\usepackage{psfig}
%\usepackage{epsfig}
\usepackage{graphicx}
\usepackage{amssymb}
\usepackage{multicol}
\usepackage[bottom]{footmisc}
\usepackage{subfigure}
\usepackage[OT2,OT1]{fontenc}
\newcommand{\imsize}{3in}

\useinnertheme{rounded}
\usecolortheme{whale}
%\usecolortheme{orchid}
\useoutertheme{infolines}
%\useoutertheme{shadow}

%change your title
\title{Analysis using Belief Propagation Techniques}
%\setbeamercolor{title}{fg=white,bg=black}

\begin{document}

\author[Chitra Suresh] {By \\ \vspace{0.05in} \textbf{Chitra Suresh } \\ \vspace{0.01in} {Under the Guidance of} \\ \textbf{Prof.Dr.Kushal.R.Tuckley . }\\
\textcolor{black}{ \\ Department of Electronics Engineering} \\ \textcolor{black}{Ramrao Adik Institute of Technology,\\ Nerul, Navi Mumbai } \\}

\begin{frame}
\begin{center}
\end{center}
\ Analysis using Belief Propagation Techniques

Under the Guidance of \\ \textbf{Prof.Dr.Kushal.R.Tuckley }

\end{frame}




\begin{frame}
\frametitle{What is spintronics?}
\begin{itemize}
\item {The automatic synthesis of a SISO and MIMO QFT controllers is still
an open problem.}
\item {The most successful method for such a design takes
into consideration the non-linear/non-convex QFT bounds without any
approximation.}
\item {It thereby ensures closed loop stability of the
system, and becomes largely independent of the initial controller
solution.}
\item {Magnetic Levitation system is subjected to many external
disturbances.}
\item {It is highly nonlinear system.}
\end{itemize}
\end{frame}


%Slide which includes bullated items having numerical no.
\begin{frame}
\frametitle{Outline}
\begin{enumerate}
\item {Introduction}
\item {Brief of Magnetic Levitation setup}
\item {Mathematical Modelling og Magnetic Levitation System}
\item {Preliminaries- QFT and Constraint Solver}
\item {QFT Controller Synthesis Problem}
\item {Proposed QFT Controller Synthesis Method and Prefilter Design  for SISO case}
\item {Proposed QFT Controller Synthesis Method and Prefilter Design  for MIMO case}
\item {Discussion}
\item {Conclusions and Future Work}
\end{enumerate}
\end{frame}

%Slide which includes figure
%\begin{frame}
%\frametitle{Introduction to Magnetic Levitation Setup} Schematic of




%Slide which includes table
\begin{frame}
\frametitle{Sensor Linearization for Lower Magnet and Coil}
\begin{table}
  \centering
  \caption{Raw Sensor Data for Lower SISO case}
  \begin{tabular}{|c|c|}
\hline
    \textbf{Magnet position (cm)} & \textbf{Raw Sensor Output $y_{1raw}$(counts)} \\ \hline
    % after \\: \hline or \cline{col1-col2} \cline{col3-col4} ...
    0 & 27900 \\ \hline
    0.5 & 22700 \\ \hline
    1 & 18300 \\ \hline
    2 & 12000 \\ \hline
    3 & 8200 \\ \hline
    4 & 5800 \\ \hline
    5 & 4100 \\ \hline
    6 & 2800 \\ \hline
  \end{tabular}
\end{table}
\end{frame}

%Slide which includes equation
\begin{frame}
\frametitle{Sensor Nonlinearity}
\begin{eqnarray*}
% \nonumber to remove numbering (before each equation)
  y_{ical}&=& \frac{e_{i}}{y_{iraw}}+\frac{f_{i}}{\sqrt{y_{iraw}}}+g_{i}+h_{i}y_{iraw}\\
  e_{1} &=& -11347,f_{1}=713.5441,g_{1}=-2.9904,h_{1}=-2.5283*10^{-5} \\
  e_{2} &=& 7109.4,f_{2}=-581.75,g_{2}=1.8355,h_{2}=4.1371*10^{-5}
\end{eqnarray*}
\end{frame}


%Slide which includes last slides
\begin{frame}
\centerline{\textbf{\begin{Huge}Thank You...\end{Huge}}}
\end{frame}

\end{document}

