\chapter{\textbf{Special feature on MSPPT in research}}


\textbf{4.1. Setting time for each slide}
\begin{itemize} 
  \item \textbf{Step 1:} Open the presentation for which to specify the amount of time between slides
  \item \textbf{Step 2:} Click inside the column at the left side of the window showing slide previews, then press Ctrl + A on your keyboard to select all of them
  \item \textbf{Step 3:} Click Transitions at the top of the window
  \item \textbf{Step 4:} Click inside the box to the left of On Mouse Click, in the Timing section of the window, to clear the check mark
  \item \textbf{Step 5:} Check the box to the left of after to check the box, and then specify the amount of time for which each slide to be displayed.
\end{itemize}


We can also individually set the amount of time per slide by skipping step 2, then repeating steps 3-5 for each individual slide
\textbf{4.2. Hyper links}\cite {Codeman Lisa}
\textbf{4.2.1. Create a hyper link to a slide in the same presentation}
\begin{itemize}
  \item In Normal view, select the text or the object that you want to use as a hyper link.
  \item On the Insert tab, in the Links group, click Hyper link
  \item Under Link to, click Place in This Document.
  \item Do one of the following:
\end{itemize}
	
\textbf{4.2.3.Link to a custom show in the current presentation:}
\begin{itemize}
  \item Under select a place in this document, click the custom shows that to use as the hyper link destination.
  \item Select the Show and return check box.
\end{itemize}

\textbf{4.2.4.Link to a slide in the current presentation:}
Under Select a place in this document, click the slide that you want to use as the hyper link destination.
\textbf{4.2.5. Create a hyper link to a slide in a different presentation}
\begin{itemize}
  \item In Normal view, select the text or the object that you want to use as a hyper link.
  \item On the Insert tab, in the Links group, click Hyper link
  \item Under Link to, click Existing File or Web Page.
  \item  Locate the presentation that contains the slide that you want to link to.
  \item Click Bookmark, and then click the title of the slide that you want to link to.
\end{itemize}

\textbf{4.2.6. Create a hyper link to a new file} \cite {Anna willms web}
\begin{itemize}
  \item In Normal view, select the text or the object that you want to use as a hyper link.
  \item On the Insert tab, in the Links group, click Hyper link.
  \item Under Link to, click Create New Document.
  \item In the Name of new document box, type the name of the file that you want to create and link to.
To create a document in a different location, under Full path, click Change, browse to the location where to create the file, and then click OK.
  \item To edit, click whether to change the file now or later.
\end{itemize}


\textbf{4.3. Adding You tube video to the slide}

First download the YouTube video to computer in either Windows Media or AVI format since PowerPoint doesn't understand the default FLV or MP4 formats of You Tube. Once the video is saved as an AVI or WMV file, switch to PowerPoint and choose Insert - > Movie - > "Movie from file" to put the YouTube video into the current slide.
\textbf{4.4. Incorporating Word document /Excel into power point}
\begin{itemize}
  \item Click the Insert tab, and then click Insert Object button.
  \item  Click the Create new option, and then click Microsoft Word Document, or click the Create from file option, click the Browse button, and then 	locate and select the file you want.
  \item Click  OK.
\end{itemize}


\textbf{4.5. Incorporating MAT Lab or ORCAD into power point:}
\begin{itemize}
  \item The software which is related to file or program should be installed in the same computer where PPT is created.
  \item Open power point and select object to insert
  \item Select "create from file"(which is created in mat lib) and browse
  \item Select  file and press ok
  \item Select "display as icon". Name can be given by selecting Change icon
  \item Now  selected file is embedded in your  PPT
\end{itemize}

\textbf{4.5.Benefits/Limitations of Microsoft Power Point}
\textbf{Benefits}
\begin{itemize}
  \item The text,images ,audio and video can be combined.
  \item The lectures can be created.
  \item The videos can be created for the presentation that can play automatically.
  \item The slides can be animated to reveal the information when presenter want to.
  \item The presentations can be recorded for others to  view later.
  \item The presentation can be saved as pdf
\end{itemize}
	

\textbf{Limitations:}
\begin{itemize}
  \item Only certain audio, video and images files work well in power point.
  \item MS Power Point does not save well to a purely video format for sharing.
\end{itemize}


